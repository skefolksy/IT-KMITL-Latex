\chapter{บทนำ}
\label{chapter:introduction}

\section{แนวคิด ที่มา และความสำคัญ}
ภาษาคอมพิวเตอร์หรือ Coding เข้ามามีบทบาทสำคัญในช่วงที่เทคโนโลยีมีการเปลี่ยนแปลงอย่างก้าวกระโดด โดยเฉพาะอย่างยิ่งในส่วนของภาคธุรกิจที่เริ่มมีการนำเทคโนโลยีมาปรับใช้ทั้งภายใน และภายนอกองค์กรเพิ่มมากขึ้น
ส่งผลให้อัตราการจ้างงานในรูปแบบของบุคลากรลดลงในหลายแผนก โดยเฉพาะแผนกที่เทคโนโลยีสามารถทำงานได้มีประสิทธิภาพกว่าการใช้ทรัพยากรมนุษย์ แต่ในทางตรงกันข้ามบริษัทเบื้องหลังที่ผลิตเทคโนโลยีต่าง ๆ
กลับมีแนวโน้มในการเพิ่มบุคลากรทางคอมพิวเตอร์เพื่อพัฒนา และเพิ่มกำลังการผลิตเทคโนโลยี ให้ตอบสนองต่อความต้องการของหลายภาคส่วน ด้วยเหตุนั้น ภาษาคอมพิวเตอร์จึงเริ่มเข้ามาเป็นพื้นฐานของระบบการศึกษาของเด็กในศตวรรษที่ 21
เพื่อเสริมสร้างทักษะการคิดวิเคราะห์ และฝึกฝนกระบวนการคิดอย่างเป็นเหตุเป็นผล ซึ่งเป็นสิ่งที่จำเป็นอย่างยิ่งสำหรับการดำเนินชีวิตในปัจจุบัน และต่อไปในอนาคต

กระนั้น การเข้าถึงภาษาคอมพิวเตอร์ทางตรงยังมีข้อจำกัดหลายอย่างทั้งด้านอุปกรณ์ หรือ บุคลากรทางการศึกษาที่มีอย่างจำกัด ซึ่งทำให้การศึกษาภาษาคอมพิวเตอร์ยังคงเป็นไปได้ยาก
ทางผู้จัดทำจึงได้เล็งเห็นถึงแนวทางในการนำความรู้ด้านการคิดเชิงคำนวณ \cite{KnowComputationalThinking} (Computational Thinking \cite{IntroductionToComputationalThinking}) ที่ประกอบไปด้วย 4 แนวคิด ดังนี้
\begin{enumerate}
    \item การแบ่งย่อยปัญหา (Decomposition)
    \item การเข้าใจรูปแบบ (Pattern Recognition)
    \item ความคิดเชิงนามธรรม (Abstraction)
    \item การออกแบบขั้นตอนวิธี (Algorithm Design)
\end{enumerate}
ซึ่งเป็นพื้นฐานของการเรียนรู้ภาษาคอมพิวเตอร์มาพัฒนาเป็นสื่อการเรียนรู้ในรูปแบบของการเขียนโปรแกรมแบบจับต้องได้ เพื่อให้เด็ก และเยาวชน สามารถเรียนรู้ เข้าใจ
และเข้าถึงทักษะการคิดเชิงคำนวณ และทักษะการคิดวิเคราะห์ อีกทั้งผู้จัดทำยังต้องการที่จะลดปัญหาสายตาในเด็กที่อาจเกิดจากการทำกิจกรรมหน้าจอคอมพิวเตอร์เป็นเวลานาน
(ซึ่งหากเป็นการเรียนรู้การเขียนภาษาคอมพิวเตอร์รูปแบบเดิมส่วนใหญ่จะต้องทำผ่านคอมพิวเตอร์เป็นหลัก) โดยการพัฒนาอุปกรณ์ Microcontroller
เพื่อใช้สำหรับเป็นต้นแบบสื่อการเรียนรู้การคิดเชิงคำนวณที่เหมาะกับเด็ก และเยาวชนในรูปแบบการเขียนโปรแกรมแบบจับต้องได้ ซึ่งอุปกรณ์นี้จะไม่มีการใช้หน้าจอทุกชนิด
โดยจะทดแทนด้วยหุ่นยนต์ การ์ด และแผนที่ที่มีการออกแบบมาเพื่อเด็ก และเยาวชนโดยเฉพาะ นอกจากนั้นยังอาศัยอุปกรณ์ที่มีราคาไม่แพง รวมถึงมีวิธีการสร้างที่ไม่ซับซ้อนซึ่งเหมาะกับการผลิตเพื่อใช้งานเองของโรงเรียน องค์กร หรือบุคคลทั่วไป

\section{วัตถุประสงค์ของโครงงาน}
\begin{enumerate}
    \item พัฒนาสื่อการเรียนรู้ทักษะการคิดวิเคราะห์ และการแก้ปัญหา ในรูปแบบของการเขียนโปรแกรมแบบจับต้องได้ ที่เหมาะสมสำหรับเด็ก และเยาวชนในแต่ละช่วงอายุ
    \item ส่งเสริมการเรียนการสอนทักษะการคิดวิเคราะห์ และการแก้ปัญหา โดยเริ่มตั้งแต่ระดับประถมศึกษาขึ้นไป 
    \item สร้างแรงจูงใจแก่เด็ก และเยาวชน ในการพัฒนาทักษะการคิดวิเคราะห์ และการแก้ปัญหาอย่างเป็นขั้นตอน
\end{enumerate}

\section{ขอบเขตการดำเนินงาน}
พัฒนาอุปกรณ์ Microcontroller เพื่อใช้สำหรับเป็นสื่อการเรียนรู้การคิดเชิงคำณวณ สำหรับเด็ก และเยาวชนที่อายุตั้งแต่ 7 ปีขึ้นไป ด้วยอุปกรณ์ที่สามารถจัดหาได้ง่าย ราคาถูก
โดยจะทำให้ชิ้นส่วนอื่นที่นอกเหนือจากตัว Microcontroller มีลักษณะที่เข้าใจง่าย และส่งเสริมพัฒนาการด้านการคิดวิเคราะห์ของเด็กโดยเฉพาะ ประกอบไปด้วย
การ์ดที่อยู่ในรูปแบบของจิ๊กซอว์ และแผนที่ตาราง รวมไปถึงการจัดทำคู่มือผู้ใช้ และวิธีการติดตั้งที่เข้าใจง่ายรองรับการเป็น Opensource สำหรับผู้ปกครอง หรือโรงเรียนที่สนใจนำไปเป็นสื่อการเรียนรู้ให้เด็ก และเยาวชน

\section{ขั้นตอนการดำเนินงาน}
\begin{enumerate}
    \item กำหนด และวิเคราะห์ปัญหา
    \item ศึกษาทฤษฎี และหลักการของการคิดเชิงคำนวณ
    \item ศึกษาพัฒนาการด้านการเรียนรู้ของเด็ก และเยาวชนที่มีอายุตั้งแต่ 7 ปีขึ้นไป
    \item ศึกษาอุปกรณ์ และเทคโนโลยีที่เกี่ยวข้องกับการพัฒนาสื่อการเรียนรู้สำหรับเด็ก
    \item ออกแบบสื่อการเรียนรู้
    \item พัฒนาต้นแบบสื่อการเรียนรู้
    \item ทดสอบ และปรับปรุง
    \item จัดทำคู่มือสำหรับผู้ใช้
    \item สรุปผลการพัฒนา
    
\end{enumerate}

\section{ประโยชน์ที่คาดว่าจะได้รับ}
\begin{enumerate}
    \item เพิ่มประสิทธิภาพการเรียนรู้ทักษะการคิดเชิงคำนวณ ซึ่งเป็นพื้นฐานของการเขียนโปรแกรม ให้แก่เด็ก และเยาวชน
    \item เด็ก และเยาวชน สามาถพัฒนาทักษะการคิดเชิงคำนวณได้อย่างทั่วถึง
    \item สนับสนุนเด็ก เยาวชน และบุคคลทั่วไป ที่มีความสนใจ และต้องการฝึกฝนทักษะการคิดเชิงคำนวณ
    
\end{enumerate}